\documentclass[danish]{../resume}
\usepackage{xcolor}
\definecolor{bronze}{rgb}{0.8, 0.5, 0.2}
\definecolor{gold}{rgb}{1.0, 0.84, 0.0}

\begin{document}
\fontfamily{ppl}\selectfont
\noindent 
\begin{tabularx}{\linewidth}{@{}m{0.75\textwidth} m{0\textwidth}@{}}
{
    \large {Markus Engelund Mathiasen} \newline
    \small{
        \clink{
            Email: \href{mailto:markusm@cs.au.dk}{markusm@cs.au.dk} \newline
             \href{https://github.com/markusmathiasen/}{GitHub}\hspace{2em}
             \href{https://www.linkedin.com/in/markus-mathiasen-5329a4303/}{LinkedIn}
        } \newline
    }
}
& 
{
    \includegraphics[height=4cm]{../markus}
}
\end{tabularx}

\csection{Resumé}{\small
    \begin{itemize}
        \item[] Jeg er en ambitiøs studerende, og er i gang med min PhD i datalogi ved Aarhus Universitet. Mit arbejde handler om at opnå bedre teoretisk forståelse af klassiske machine learning algoritmer. I løbet af min bachelor har jeg lavet ekstra aktiviteter i et program kaldet Talent Track, hvor jeg fik mulighed for at indgå i spændende projekter i de forskellige forskningsgrupper på datalogi. I min fritid kan jeg godt lide at lave konkurrenceprogrammering, hvor jeg skærper mine evner ved at løse og implementere udfordrende algoritmiske problemer.
    \end{itemize}
}
\csection{Uddannelse}{\small
    \begin{itemize}
        \item \textbf{PhD i datalogi (igangværende)\hfill{(2024-2028)}}\newline
        Aarhus Universitet\\
        \textit{Projekt titel:} Theoretical Understanding of Classic Learning Algorithms\\
        \textit{Vejleder:} Kasper Green Larsen
        \item \textbf{Kandidat i datalogi (igangværende)\hfill{(2023-2026)}}\newline
        Aarhus Universitet
        
        \item \textbf{Bachelor i datalogi\hfill(2020-2023)}\newline
        Aarhus Universitet\\
        \underline{Bachelor projekt:}\\
        \textit{Titel:} Graph Connectivity in the Semi-Streaming Model\\
        \textit{Vejleder:} Chris Schwiegelshohn\\
        I dette projekt præsenterede jeg kendte resultater om, hvordan man kan afgøre, om en graf er forbundet i semi-streaming modellen. Jeg undersøgte også, hvad vi kan gøre, hvis vi gerne vil gøre algoritmen robust mod en adaptiv adversary.
    \end{itemize}
}

\csection{Publikationer}{\small
    \begin{itemize}
        \item \textbf{The Many Faces of Optimal Weak-to-Strong Learning\hfill(2024)}\\
        \textit{Med:} Mikael Møller Høgsgaard og Kasper Green Larsen\\
        \textit{Konference:} Neural Information Processing Systems 2024 (NeurIPS)
        \item \textbf{Improved Replicable Boosting with Majority-of-Majorities\hfill(20??)}\\
        \textit{Med:} Kasper Green Larsen og Clement Svendsen\\
        \textit{Konference:} Under review.
    \end{itemize}
}

\csection{Undervisningserfarring}{\small
    \begin{itemize}
        \item \textbf{Instruktor\hfill(2021-2025)}\newline
        Jeg har været ansat som instruktor ved Aarhus Universitet i følgende bachelor aktiviteter på datalogi.
        \begin{itemize}
            \item Algoritmer og data strukturer\hfill(2 semestre)
            \item Machine learning\hfill(1 semester)
            \item Nummerisk lineær algebra\hfill(1 semester)
            \item Programmeringscafé\hfill(4 semestre)
            \item Programmeringssprog\hfill(2 semestre)
        \end{itemize}
        \item \textbf{Underviser ved Dansk Datalogi Dyst\hfill(2021)}\newline
        Jeg har lavet øvelser, problemer og undervist ved "Dansk Datalogi Dyst", hvor vi træner og udtager de bedste gymnasieelever til at repræsentere Danmark ved den internationale olympiade i informatik (IOI).
    \end{itemize}
}

\csection{Talent Track}{\small
    \begin{itemize}
        \item[] Talent Track er en mulighed for bachelorstuderende med et gennemsnit på mindst 10 til at lave ektra aktiviteter i de forskellige forskningsgrupper på datalogi ud over deres normale studier. Jeg har lavet ekstra arbejde svarende til 30 ECTS i løbet af min bachelor. Jeg lavede følgende projekter de sidste fire semestre af min bachelor:
        \item \textbf{Oblivious Data Structures\hfill(Forår 2023)}\newline
        Projekt med Peter Scholl fra kryptologi gruppen, hvor vi kiggede på forskellige datastrukturer og hvordan man enten kan gøre dem oblivious ved at bruge oblivious prioritetskøer, eller vise at det er et svært problem ved at reducere det til ORAM.
        \item \textbf{Minimum DNF with Xor\hfill(Efterår 2022)}\newline
        Projekt med Srikanth Srinivasan fra kompleksitetsgruppen, hvor vi kiggede på en variation af minimum DNF problemet, der tillader xor imellem litterals. Mere koknkret kiggede vi på hvorfor reduktionen til set cover ikke virker i denne variation og afprøvede alternative tilgange til at konstruere en approksimationsalgoritme til dette problem.
        \item \textbf{Micro WebAssembly in Coq\hfill(Forår 2022)}\newline
        Projekt med Jean Yves Alexis Pichon fra logik og semantik gruppen, hvor vi kiggede på at definere webassembly i Coq, konstruere et syntaktisk og semantisk typesystem for dette, og bevise deres relation.
        \item \textbf{Randomized Algorithms\hfill(Efterår 2021)}\newline
        Projekt med Kasper Green Larsen fra algoritme gruppen, hvor vi kiggede på count sketch og hvordan den kan blive brugt til at finde heavy hitters i en datastream effektivt.
    \end{itemize}
}
    
\csection{Konkurrenceprogrammering}{\small
    \begin{itemize}
        \item \textbf{DM i programmering\hfill(2024)}\newline
        \textcolor{gold}{\textbf{Vandt}} danmarksmesterskabet i programmering for studerende.
        \item \textbf{Northwestern Europe Regional Contest (NWERC)\hfill(2020-2024)}\newline
        Kvalificerede til Aarhus Universitets hold 5 gange og vandt \textcolor{gold}{\textbf{guldmedalje}} in 2024.
        \item \textbf{Google Code Jam\hfill(2022)}\newline
        Kom til runde 3 (sidste runde før verdensfinalen) og blev nummer 435 i verden ud af 32.000 deltagere i alt.
        \item \textbf{International Olympiade i Informatik (IOI)\hfill(2020)}\newline
        Kvalificerede til det danske hold.
        \item \textbf{Baltisk Olympiade i Informatik (BOI)\hfill(2020)}\newline
        Kvalificerede til det danske hold og vandt \textcolor{bronze}{\textbf{bronzemedalje}}.
    \end{itemize}
}

\end{document}